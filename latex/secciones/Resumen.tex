
El Gobierno del Estado de Yucatán tiene como meta construir una entidad competitiva, incluyente
y equitativa en todos los ámbitos. Ello presupone, en primer lugar, lograr un crecimiento
económico sin menoscabo del capital natural y los servicios ambientales con los que cuenta el
estado para el bienestar de sus habitantes. Conforme se señala en el Plan Estatal de Desarrollo
2012-2018, esto implica optimizar el funcionamiento de los instrumentos de planeación y política
ambiental, particularmente el ordenamiento ecológico del territorio. De acuerdo con la legislación
vigente, lo anterior conlleva asumir el reto de lograr una gestión pública armónica y diligente. Ésta
debe manifestarse en una coordinación efectiva entre las entidades de gobierno y los diferentes
actores sociales que intervienen en la formulación y ejecución de múltiples programas y proyectos
de desarrollo. Al encauzar la política pública bajo tales lineamientos, se establece tácitamente el
compromiso de la innovación para el desarrollo sostenible en Yucatán.  En este contexto, la
demanda específica para el “Establecimiento de un Sistema de Gestión para el Ordenamiento
Ecológico del Territorio del Estado de Yucatán” resulta por demás oportuna. Por consiguiente, la
propuesta que presenta el Laboratorio Nacional de Ciencias de la Sostenibilidad (LANCIS) tiene
como propósito desarrollar los conocimientos y capacidades, ampliar la infraestructura e impulsar
las innovaciones que permitan al Gobierno de Yucatán liderar la transición al desarrollo sostenible
de la región. \\

Para lograrlo, el LANCIS ha propuesto un diseñar, desarrollar e implementar un sistema de gestión para el ordenamiento ecológico del
territorio del estado de Yucatán (Sistema) que funcione como un sistema de información
geográfica para la puesta en marcha de la actualización del POETY, para el manejo, análisis y
visualización de información que facilite la gobernanza colaborativa en el proceso de
ordenamiento ecológico en la entidad y su articulación con otros instrumentos de planeación
pertinentes.\\

Este documento contiene requerimientos funcionales y arquitectónicos del diseño de dicho sistema. El primer capítulo aborda los requerimientos del sistema, es decir describe los procesos
y capacidades de los usuarios dentro este. El segundo capítulo describe la arquitectura del sistema, esto es, los componentes de software que lo componen y la forma como se integran.\\